\documentclass[11pt]{article}

\usepackage[T1]{fontenc}
\usepackage[utf8]{inputenc}
\usepackage[francais]{babel}
\usepackage{fancyhdr}
\usepackage{datetime} % access to \currenttime
\usepackage{hyperref}
\usepackage{tikz}

\setlength{\parskip}{2ex}

\title{\vspace{-5em}
KAP ZIK \\
\vspace{1em}
Un jeu pour travailler la musique et le solfège}
\author{par Christophe Gragnic}
\date{} 

\pagestyle{fancy}
\fancyhead[R]{\today~-~\currenttime}
\fancyfoot[C]{\thepage}

\begin{document}

\maketitle

\setcounter{tocdepth}{2}
\tableofcontents

\section{Introduction}

Voici un jeu qui vous fera travailler le nom des notes, la construction des
gammes, des accords…

Pour l’instant, seule la mécanique du jeu est décrite, il n’y a pas encore de
thème, pas d’illustrations non plus.

Voir la section \ref{contact} à la fin de ce document si vous voulez partager
vos idées.

\newpage

\section{Je veux jouer tout de suite !}

Avec plaisir ! C’est le meilleur moyen d’apprendre à jouer.

\subsection{Préparation}

\begin{itemize}
\item Si ce n’est pas encore fait, découpez les cartes.
\item Regroupez et mélangez les cartes de tirage au sort :
    \begin{itemize}
    \item noms de notes
    \item degrés
    \end{itemize}
\item Vous pouvez commencer !
\end{itemize}

\subsection{Un exemple de partie}

Le premier parcours à essayer est le parcours « Débutant 1 », vous le trouverez
sur une des fiches du jeu mais nous vous le montrons ci-dessous :

\begin{tikzpicture}
\node[draw] (A) at (0,0) {AVO};
\node[draw] (M) at (2,0) {MAK};
\node[draw] (K) at (4,1) {KAP};
\node[draw] (Z) at (4,-1) {ZIK};
\draw (A) -- (M);
\draw (M) -- (K);
\draw (M) -- (Z);
\end{tikzpicture}

\section{Idées à exploiter}

\begin{itemize}
\item Demi-tons entre Mi et Fa, et entre Si et Do.
\item Clefs (Sol, Fa…)
\item Notes sur le clavier, sur la guitare, sur une portée.
\item Rythme, chiffrage des mesures.
\item Cadences.
\item Sciences physiques de la musique (vibrations, fréquences…).
\item Exécution de la musique : taper dans les mains, chanter, jouer un
    instrument.
\end{itemize}

\section{Contact} \label{contact}

\setlength{\parindent}{0pt}

Site~: \url{http://profgra.org/kapzik/}

Email~: \url{profgra.org@gmail.com}

\end{document}
