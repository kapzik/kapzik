\documentclass[11pt]{article}

\usepackage[T1]{fontenc}
\usepackage[utf8]{inputenc}
\usepackage[francais]{babel}
\usepackage{fullpage}
\usepackage{fancyhdr}
\usepackage{datetime} % access to \currenttime
\usepackage{hyperref}
\usepackage{tikz}
\usetikzlibrary{shapes}

\setlength{\parskip}{2ex}

\title{\vspace{-5em}
KAP ZIK \\
\vspace{1em}
Un jeu pour travailler la musique et le solfège}
\author{par Christophe Gragnic}
\date{} 

\pagestyle{fancy}
\fancyhead[R]{\today~-~\currenttime}
\fancyfoot[C]{\thepage}

% Mes commandes
\newcommand{\B}[0]
{$\flat$}

\newcommand{\D}[0]
{$\sharp$}

\newcommand{\carte}[1]
{\textbf{#1}}

\newcommand{\parcours}[1]
{\textbf{#1}}

\newcommand{\QST}[1]
{\textbf{#1}}

\newcommand{\mode}[1]
{\textbf{#1}}

\newcommand{\horlogenoms}[0]
{
\begin{tikzpicture}
\node[draw] (C) at (   0,  0)   {Do};
\node[draw] (D) at ( 1.5, -0.7) {Ré};
\node[draw] (E) at ( 1.8, -2)   {Mi};
\node[draw] (F) at ( 0.8, -3.2) {Fa};
\node[draw] (G) at (-0.8, -3.2) {Sol};
\node[draw] (A) at (-1.8, -2)   {La};
\node[draw] (B) at (-1.5, -0.7) {Si};
\draw (C) -- (D) -- (E) -- (F) -- (G) -- (A) -- (B) -- (C);
\end{tikzpicture}
}

\newcommand{\horlogenotes}[0]
{
\begin{tikzpicture}
\node[draw] (C)  at (   0, -0.2)   {Do};
\node[draw] (Db) at ( 1.7, -0.7) {Do \D/Ré \B};
\node[draw] (D)  at ( 2.3, -1.6) {Ré};
\node[draw] (Eb) at ( 2.6, -2.45) {Ré \D/Mi \B};
\node[draw] (E)  at ( 2.3, -3.3) {Mi};
\node[draw] (F)  at ( 1.4, -4) {Fa};
\node[draw] (Gb) at (   0, -4.5) {Fa \D/Sol \B};
\node[draw] (G)  at (-1.4, -4) {Sol};
\node[draw] (Ab) at (-2.3, -3.3) {Sol \D/La \B};
\node[draw] (A)  at (-2.6, -2.45) {La};
\node[draw] (Bb) at (-2.3, -1.6) {La \D/Si \B};
\node[draw] (B)  at (-1.4, -0.7) {Si};
\draw (C) -- (Db) -- (D)  -- (Eb) -- (E)  -- (F) -- (Gb)
          -- (G)  -- (Ab) -- (A)  -- (Bb) -- (B) -- (C);
\end{tikzpicture}
}

\newcommand{\scores}[0]
{
\begin{tabular}{ | *{7}{p{0.1cm}cp{0.1cm} |} }
    \hline
    & Date : &&&&&& Parc. : \\
    \hline
    & Qst &&& J1 &&& J2 &&& J3 &&& J4 &&& J5 &&& J6 & \\
    \hline
    &&&&&&&&&&&&&&&&&&&& \\
    \hline
    &&&&&&&&&&&&&&&&&&&& \\
    \hline
    &&&&&&&&&&&&&&&&&&&& \\
    \hline
    &&&&&&&&&&&&&&&&&&&& \\
    \hline
    &&&&&&&&&&&&&&&&&&&& \\
    \hline
    &&&&&&&&&&&&&&&&&&&& \\
    \hline
    &&&&&&&&&&&&&&&&&&&& \\
    \hline
\end{tabular}
}

\begin{document}

\maketitle

\setcounter{tocdepth}{2}
\tableofcontents

\section{Introduction}

Voici un jeu qui vous fera travailler le nom des notes, la construction des
gammes, des accords…

Pour l’instant, seule la mécanique du jeu est décrite, il n’y a pas encore de
thème, pas d’illustrations non plus.

Voir la section \ref{contact} à la fin de ce document si vous voulez partager
vos idées.

\section{Je veux jouer tout de suite !}

Avec plaisir ! C’est le meilleur moyen d’apprendre à jouer.

\subsection{Préparation}

Attention, avec ce premier prototype il n’y a rien à découper.

Pour les questions, il faut se référer à la liste en section
\ref{liste questions}, et pour les fiches d’aide, c’est en section
\ref{aide}.

Pour les tirages au sort, mieux vaut fabriquer des cartes en découpant des
feuilles de brouillon. Vous aurez besoin des cartes :

\begin{itemize}
\item noms de notes
\item degrés
\item notes (inutiles dans les parcours pour débutants)
\item nombre de bémols ou de dièses (inutiles dans les parcours pour débutants)
\end{itemize}

Vous pouvez donc ignorer ce qui suit et passer à la section suivante !

\begin{itemize}
\item Si ce n’est pas encore fait, découpez toutes les cartes.
\item Regroupez les cartes « question ». Il sera bien pratique de les laisser
    dans l’ordre alphabétique de leur code.
\item Regroupez et mélangez les cartes de tirage au sort :
    \begin{itemize}
    \item noms de notes
    \item noms de notes anglos-saxons
    \item degrés
    \item notes
    \item nombre de bémols ou de dièses
    \end{itemize}
\item Placez les cartes « Aide » à disposition des joueurs :
    \begin{itemize}
    \item horloge des noms de notes
    \item harmonisation de la gamme majeure
    \end{itemize}
\item Vous pouvez commencer !
\end{itemize}

\subsection{Un exemple de partie}

Une fois que vous aurez fait quelques parties de ce type, vous voudrez utiliser
des règles plus sophistiquées. Vous les trouverez dans la partie \ref{modes}.

\subsubsection{Choix du parcours}

Le premier parcours à essayer est le parcours \parcours{Débutant 1}, en haut
sur cette figure :

\begin{tikzpicture}
\node[draw] (A) at (0, 0) {AVO};
\node[draw] (M) at (2, 0) {MAK};
\node[draw] (K) at (4, 1) {KAP};
\node[draw] (Z) at (6, 1) {ZIK};
\node[draw,minimum height=6mm] (X) at (4,-1) {...};
\node[draw,minimum height=6mm] (Y) at (6,-1) {...};
\node[draw,rounded corners=3pt] (D1) at (8.5, 1) {Débutant 1};
\node[draw,minimum height=6mm, rounded corners=3pt] (D2) at (8.5,-1) {...};
\draw[->] (A) -- (M);
\draw[->] (M) -- (K);
\draw[->] (K) -- (Z);
\draw[->] (M) -- (X);
\draw[->] (X) -- (Y);
\end{tikzpicture}

Les pointillés correspondent à un autre parcours. Vous trouverez les autres
parcours dans la section \ref{parcours} plus loin dans cette notice.

La première question de ce parcours est celle qui se trouve le plus à gauche.
C’est donc la question \QST{AVO}. C’est un code qui ne veut rien dire et sert
seulement à retrouver la question. Retrouvez-là dans le paquet des questions
et posez-là face visible devant les joueurs.

Le mode de jeu le plus simple est le \mode{mode collaboratif} : les joueurs
doivent finir le parcours en moins de 10 minutes.

C’est donc le moment de démarrer un compte à rebours de 10 minutes !

\subsubsection{Quelques tours de jeu}

La première question, codée \QST{AVO}, est :

\begin{quote}
Tirez une carte \carte{nom de note} et donnez la note suivante dans l’horloge
des noms de notes. Exemple : Si vous tirez « Fa » il faut dire « Sol ».
\end{quote}

Pour chaque joueur, il faudra tirer une carte \carte{nom de note} et poser par
exemple la question (si le nom de note tiré est « Fa ») : « Quelle est la note
qui suit Fa ? ». Avec toutes les informations fournies par les cartes
« Aide », le joueur finira bien par donner la bonne réponse, ici Sol, et on
passera au joueur suivant. Vous pouvez remettre la carte tirée sous son tas.

Un tour de jeu avec la question \QST{AVO} consiste à utiliser cette question
pour chaque joueur. Suivant les tirages, ils auront donc peut-être des
questions différentes !

Après trois tours de jeu, on passe à la question suivante. Dans ce parcours,
c’est la question \QST{MAK} qui vient après la question \QST{AVO}.

\subsubsection{Fin de la partie}

Dans le mode \mode{collaboratif}, la partie se termine dans un des deux cas
suivants :

\begin{itemize}
    \item le compte à rebours est terminé (les 10 minutes sont écoulées) ;
    \item il y a eu trois tours de jeu avec la dernière question du parcours
        (pour le parcours \parcours{Débutant 1}, c’est la question \QST{ZIK}).
\end{itemize}

\subsection{Après la première partie}

Une fois que vous aurez fait quelques parties de ce type, les joueurs les plus
en confiance pourront jouer sans les fiches d’aide.

De plus, vous voudrez peut-être utiliser des règles plus sophistiquées. Vous
les trouverez dans la partie \ref{modes}.

\section{Modes de jeu} \label{modes}

\subsection{Mode collaboratif}

C’est le mode « contre la montre » expliqué dans la partie donnée en exemple.
Les joueurs ont tous gagné s’ils finissent ensemble le parcours qu’ils ont
choisi en moins de 10 minutes (ou toute autre durée convenue à l’avance).

\subsection{Mode avec handicap}

En cours d’élaboration.

Les réponses sont chronométrées. Pour gagner, un joueur devra s’être davantage
amélioré que les autres joueurs.

\subsection{Mode combat}

Les réponses sont chronométrées. Pour gagner, un joueur devra avoir
globalement répondu plus vite que les autres joueurs.

Le plus simple est d’utiliser autant de chronomètres que de joueurs
(disponibles par exemple sur un téléphone portable ou un site web). Il suffit
de ne pas remettre les chronométres à zéro entre les questions et de les
comparer à la fin du parcours.

Si vous n’avez pas autant de chronomètres que de joueurs il vous faudra
reporter le temps passé sur chaque question dans la feuille de score.

Dans tous les cas, si vous jouez avec par exemple trois tours par question, il
sera plus simple d’enchaîner les trois questions avec un même joueur.

\section{Liste des parcours de questions} \label{parcours}

\subsection{Niveau Débutant}

\begin{tikzpicture}
\node[draw] (A) at (0, 0) {AVO};
\node[draw] (M) at (2, 0) {MAK};
\node[draw] (K) at (4, 1) {KAP};
\node[draw] (Z) at (6, 1) {ZIK};
\node[draw] (B) at (4,-1) {ALE};
\node[draw] (C) at (6,-1) {OLI};
\node[draw,rounded corners=3pt] (D1) at (8.5, 1) {Débutant 1};
\node[draw,rounded corners=3pt] (D2) at (8.5,-1) {Débutant 2};
\draw[->] (A) -- (M);
\draw[->] (M) -- (K);
\draw[->] (K) -- (Z);
\draw[->] (M) -- (B);
\draw[->] (B) -- (C);
\end{tikzpicture}

\subsection{Niveau Intermédiaire}

Il n’y a pas encore de parcours pour le niveau intermédiaire.

\subsection{Niveau Expert}

Il n’y a pas encore de parcours pour les experts.

\section{Questions}

Les questions sont identifiées par un code à trois lettres.

\subsection{Prononciation des codes}

\begin{itemize}
    \item E se prononce « é » ;
    \item S se prononce « ss » ;
    \item U se prononce « ou » ;
    \item X se prononce « ch ».
\end{itemize}

Par exemple, \QST{EXU} se prononce « échou ».

\subsection{Liste des questions} \label{liste questions}

\begin{tabular}{ c p{12cm} }
    \QST{ALE} & Tirez une carte \carte{nom de note} et considérez que c’est la
                fondamentale. Répétez la fondamentale et donnez sa tierce (pas
                la note suivante mais celle d’après. Exemple : Si vous tirez
                « Fa » il faut dire « Fa, La ». \\
    \QST{AVO} & Tirez une carte \carte{nom de note} et donnez la note suivante
                dans l’horloge des noms de notes. Exemple : Si vous tirez
                « Fa » il faut dire « Sol ». \\
    \QST{KAP} & Tirez une carte \carte{nom de note}, citez cette note et donnez
                les sept notes suivantes dans l’horloge des noms de notes.
                Exemple : Si vous tirez « Fa » il faut dire « Fa, Sol, La, Si,
                Do, Ré, Mi, Fa ». \\
    \QST{MAK} & Tirez une carte \carte{nom de note} et donnez la note
                précédente dans l’horloge des noms de notes. Exemple : Si vous
                tirez « Fa » il faut dire « Mi ». \\
    \QST{OLI} & Tirez une carte \carte{nom de note} et considérez que c’est la
                fondamentale. Répétez la fondamentale et donnez la tierce et
                la quinte. Exemple : Si vous tirez « Fa » il faut dire « Fa,
                La, Do ». \\
    \QST{ZIK} & Tirez une carte \carte{nom de note}, citez cette note et donnez
                les sept notes précédentes dans l’horloge des noms de notes.
                Exemple : Si vous tirez « Fa » il faut dire « Fa, Mi, Ré, Do,
                Si, La, Sol, Fa ».
\end{tabular}

\section{Idées à exploiter}

Liste des idées à incorporer dans le jeu ou les questions.

\begin{itemize}
\item Demi-tons entre Mi et Fa, et entre Si et Do.
\item Harmonisation de la gamme majeure (avec tierces, triades, accords à 4
    sons…)
\item Clefs (Sol, Fa…)
\item Notes sur le clavier, sur la guitare, sur une portée.
\item Rythme, chiffrage des mesures.
\item Relatifs mineurs/majeurs, modes, mineur harmo, mélo…
\item Cadences.
\item Sciences physiques de la musique (vibrations, fréquences…).
\item Exécution de la musique : taper dans les mains, chanter, jouer un
    instrument.
\item Parler des extensions.
\item Laisser créer ses questions et parcours.
\end{itemize}

\section{Contact} \label{contact}

\setlength{\parindent}{0pt}

Site~: \url{http://profgra.org/kapzik/}

Email~: \url{profgra.org@gmail.com}

\newpage

\section{Feuilles de score}

Exemple pour le parcours \parcours{Débutants 1} :

\begin{tabular}{ | *{7}{p{0.1cm}cp{0.1cm} |} }
    \hline
    & Date : &&&&&& Parc. : \\
    \hline
    & Qst &&& J1 &&& J2 &&& J3 &&& J4 &&& J5 &&& J6 & \\
    \hline
    & AVO &&&&&&&&&&&&&&&&&&& \\
    \hline
    & MAK &&&&&&&&&&&&&&&&&&& \\
    \hline
    & KAP &&&&&&&&&&&&&&&&&&& \\
    \hline
    & ZIK &&&&&&&&&&&&&&&&&&& \\
    \hline
\end{tabular}

\scores

\scores

\scores

\newpage

\section{Éléments à découper}

\subsection{Fiches Aide} \label{aide}

\horlogenoms
\hspace{1cm}
\horlogenotes

\vspace{1cm}

\begin{tabular}{ | *{9}{c|} }
    \hline
    1 & 2 & 3 & 4 & 5 & 6 & 7 & 1 & \D/\B \\
    \hline
    Do & Ré & Mi & Fa & Sol & La & Si & Do & 0 \\
    \hline
    Ré \B & Mi \B & Fa & Sol \B & La \B & Si \B & Do & Ré \B & 5\B \\
    \hline
    Ré & Mi & Fa \D & Sol & La & Si & Do \D & Ré & 2\D \\
    \hline
    Mi \B & Fa & Sol & La \B & Si \B & Do & Ré & Mi \B & 3\B \\
    \hline
    Mi & Fa \D & Sol \D & La & Si & Do \D & Ré \D & Mi & 4\D \\
    \hline
    Fa & Sol & La & Si \B & Do & Ré & Mi & Fa & 1\B \\
    \hline
    Fa \D & Sol \D & La \D & Si & Do \D & Ré \D & Mi \D & Fa \D & 6\D \\
    \hline
    Sol \B & La \B & Si \B & Do \B & Ré \B & Mi \B & Fa & Sol \B & 6\B \\
    \hline
    Sol & La & Si & Do & Ré & Mi & Fa \D & Sol & 1\D \\
    \hline
    La \B & Si \B & Do & Ré \B & Mi \B & Fa & Sol & La \B & 4\B \\
    \hline
    La & Si & Do \D & Ré & Mi & Fa \D & Sol \D & La & 3\D \\
    \hline
    Si \B & Do & Ré & Mi \B & Fa & Sol & La & Si \B & 2\B \\
    \hline
    Si & Do \D & Ré \D & Mi & Fa \D & Sol \D & La \D & Si & 5\D \\
    \hline
\end{tabular}

\newpage

\subsection{Cartes pour les tirages au sort}

\subsubsection{Noms de notes}

\begin{tabular}{ | *{7}{p{0.1cm}cp{0.1cm} |} }
    \hline
    & Do &&& Ré &&& Mi &&& Fa &&& Sol &&& La &&& Si & \\
    \hline
    & Do &&& Ré &&& Mi &&& Fa &&& Sol &&& La &&& Si & \\
    \hline
\end{tabular}

\subsubsection{Degrés}

\begin{tabular}{ | *{6}{p{0.1cm}cp{0.1cm} |} }
    \hline
    & 2nde &&& 3ce &&& 4te &&& 5te &&& 6te &&& 7ème & \\
    \hline
    & 2nde &&& 3ce &&& 4te &&& 5te &&& 6te &&& 7ème & \\
    \hline
    & 9ème &&& 9ème &&& 11ème &&& 11ème &&& 13ème &&& 13ème & \\
    \hline
\end{tabular}

\newpage

\end{document}
